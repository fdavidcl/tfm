\section{Add weight decay to any
autoencoder}\label{add-weight-decay-to-any-autoencoder}

Source:
\href{https://github.com/fdavidcl/ruta/blob/master/R/autoencoder_weight_decay.R}{\texttt{R/autoencoder\_weight\_decay.R}}

\texttt{add\_weight\_decay.Rd}

Adds a weight decay regularization to the encoding layer of a given
autoencoder

\begin{minted}[frame=none]{R}
add_weight_decay(learner, decay = 0.02)
\end{minted}

\hypertarget{arguments}{\subsection{\texorpdfstring{\protect\hyperlink{arguments}{}Arguments}{Arguments}}\label{arguments}}

\begin{longtable}[c]{@{}>{\small}p{3cm}>{\raggedright}p{12.5cm}@{}}
\toprule
learner & The \texttt{``ruta\_autoencoder''} object\tabularnewline
decay & Numeric value indicating the amount of decay\tabularnewline
\bottomrule
\end{longtable}

\hypertarget{value}{\subsection{\texorpdfstring{\protect\hyperlink{value}{}Value}{Value}}\label{value}}

An autoencoder object which contains the weight decay

\section{Apply filters}\label{apply-filters}

Source:
\href{https://github.com/fdavidcl/ruta/blob/master/R/filter.R}{\texttt{R/filter.R}},
\href{https://github.com/fdavidcl/ruta/blob/master/R/generics.R}{\texttt{R/generics.R}}

\texttt{apply\_filter.Rd}

Apply a filter to input data, generally a noise filter in order to train
a denoising autoencoder. Users won't generally need to use these
functions

\begin{minted}[frame=none]{R}
# S3 method for ruta_noise_zeros
apply_filter(filter, data, ...)

# S3 method for ruta_noise_ones
apply_filter(filter, data, ...)

# S3 method for ruta_noise_saltpepper
apply_filter(filter, data, ...)

# S3 method for ruta_noise_gaussian
apply_filter(filter, data, ...)

# S3 method for ruta_noise_cauchy
apply_filter(filter, data, ...)

apply_filter(filter, data, ...)
\end{minted}

\hypertarget{arguments}{\subsection{\texorpdfstring{\protect\hyperlink{arguments}{}Arguments}{Arguments}}\label{arguments}}

\begin{longtable}[c]{@{}>{\small}p{3cm}>{\raggedright}p{12.5cm}@{}}
\toprule
filter & Filter object to be applied\tabularnewline
data & Input data to be filtered\tabularnewline
\ldots{} & Other parameters\tabularnewline
\bottomrule
\end{longtable}

\hypertarget{see-also}{\subsection{\texorpdfstring{\protect\hyperlink{see-also}{}See
also}{See also}}\label{see-also}}

\texttt{autoencoder\_denoising}

\section{Coercion to ruta\_loss}\label{coercion-to-rutaux5floss}

Source:
\href{https://github.com/fdavidcl/ruta/blob/master/R/generics.R}{\texttt{R/generics.R}},
\href{https://github.com/fdavidcl/ruta/blob/master/R/loss.R}{\texttt{R/loss.R}}

\texttt{as\_loss.Rd}

Generic function to coerce objects into loss objects.

\begin{minted}[frame=none]{R}
as_loss(x)

# S3 method for character
as_loss(x)

# S3 method for ruta_loss
as_loss(x)
\end{minted}

\hypertarget{arguments}{\subsection{\texorpdfstring{\protect\hyperlink{arguments}{}Arguments}{Arguments}}\label{arguments}}

\begin{longtable}[c]{@{}>{\small}p{3cm}>{\raggedright}p{12.5cm}@{}}
\toprule
x & Object to be converted into a loss\tabularnewline
\bottomrule
\end{longtable}

\hypertarget{value}{\subsection{\texorpdfstring{\protect\hyperlink{value}{}Value}{Value}}\label{value}}

A \texttt{``ruta\_loss''} construct

\section{Coercion to ruta\_network}\label{coercion-to-rutaux5fnetwork}

Source:
\href{https://github.com/fdavidcl/ruta/blob/master/R/generics.R}{\texttt{R/generics.R}},
\href{https://github.com/fdavidcl/ruta/blob/master/R/layers.R}{\texttt{R/layers.R}},
\href{https://github.com/fdavidcl/ruta/blob/master/R/network.R}{\texttt{R/network.R}}

\texttt{as\_network.Rd}

Generic function to coerce objects into networks.

\begin{minted}[frame=none]{R}
as_network(x)

# S3 method for ruta_layer
as_network(x)

# S3 method for ruta_network
as_network(x)

# S3 method for numeric
as_network(x)

# S3 method for integer
as_network(x)
\end{minted}

\hypertarget{arguments}{\subsection{\texorpdfstring{\protect\hyperlink{arguments}{}Arguments}{Arguments}}\label{arguments}}

\begin{longtable}[c]{@{}>{\small}p{3cm}>{\raggedright}p{12.5cm}@{}}
\toprule
x & Object to be converted into a network\tabularnewline
\bottomrule
\end{longtable}

\hypertarget{value}{\subsection{\texorpdfstring{\protect\hyperlink{value}{}Value}{Value}}\label{value}}

A \texttt{``ruta\_network''} construct

\hypertarget{examples}{\subsection{\texorpdfstring{\protect\hyperlink{examples}{}Examples}{Examples}}\label{examples}}

\begin{minted}[frame=none]{R}
net <- as_network(c(784, 1000, 32))
\end{minted}

\section{Automatically compute an encoding of a data
matrix}\label{automatically-compute-an-encoding-of-a-data-matrix}

Source:
\href{https://github.com/fdavidcl/ruta/blob/master/R/autoencoder.R}{\texttt{R/autoencoder.R}}

\texttt{autoencode.Rd}

Trains an autoencoder adapted to the data and extracts its encoding for
the same data matrix.

\begin{minted}[frame=none]{R}
autoencode(data, dim, type = "basic", activation = "linear", epochs = 20)
\end{minted}

\hypertarget{arguments}{\subsection{\texorpdfstring{\protect\hyperlink{arguments}{}Arguments}{Arguments}}\label{arguments}}

\begin{longtable}[c]{@{}>{\small}p{3cm}>{\raggedright}p{12.5cm}@{}}
\toprule
data & Numeric matrix to be encoded\tabularnewline
dim & Number of variables to be used in the encoding\tabularnewline
type & Type of autoencoder to use: \texttt{``basic''}, \texttt{``sparse''},
\texttt{``contractive''}, \texttt{``denoising''}, \texttt{``robust''} or
\texttt{``variational''}\tabularnewline
activation & Activation type to be used in the encoding layer. Some
available activations are \texttt{``tanh''}, \texttt{``sigmoid''},
\texttt{``relu''}, \texttt{``elu''} and \texttt{``selu''}\tabularnewline
epochs & Number of times the data will traverse the autoencoder to
update its weights\tabularnewline
\bottomrule
\end{longtable}

\hypertarget{value}{\subsection{\texorpdfstring{\protect\hyperlink{value}{}Value}{Value}}\label{value}}

Matrix containing the encodings

\hypertarget{see-also}{\subsection{\texorpdfstring{\protect\hyperlink{see-also}{}See
also}{See also}}\label{see-also}}

\texttt{autoencoder}

\hypertarget{examples}{\subsection{\texorpdfstring{\protect\hyperlink{examples}{}Examples}{Examples}}\label{examples}}

\begin{minted}[frame=none]{R}
inputs <- as.matrix(iris[, 1:4])# Train a basic autoencoder and generate a 2-variable encoding
encoded <- autoencode(inputs, 2)

# Train a contractive autoencoder with tanh activation
encoded <- autoencode(inputs, 2, type = "contractive", activation = "tanh")
\end{minted}

\section{Create an autoencoder
learner}\label{create-an-autoencoder-learner}

Source:
\href{https://github.com/fdavidcl/ruta/blob/master/R/autoencoder.R}{\texttt{R/autoencoder.R}}

\texttt{autoencoder.Rd}

Represents a generic autoencoder network.

\begin{minted}[frame=none]{R}
autoencoder(network, loss = "mean_squared_error")
\end{minted}

\hypertarget{arguments}{\subsection{\texorpdfstring{\protect\hyperlink{arguments}{}Arguments}{Arguments}}\label{arguments}}

\begin{longtable}[c]{@{}>{\small}p{3cm}>{\raggedright}p{12.5cm}@{}}
\toprule
network & Layer construct of class \texttt{``ruta\_network''} or
coercible\tabularnewline
loss & A \texttt{``ruta\_loss''} object or a character string specifying a
loss function\tabularnewline
\bottomrule
\end{longtable}

\hypertarget{value}{\subsection{\texorpdfstring{\protect\hyperlink{value}{}Value}{Value}}\label{value}}

A construct of class \texttt{``ruta\_autoencoder''}

\hypertarget{references}{\subsection{\texorpdfstring{\protect\hyperlink{references}{}References}{References}}\label{references}}

\begin{itemize}
\tightlist
\item
  \href{https://arxiv.org/abs/1801.01586}{A practical tutorial on
  autoencoders for nonlinear feature fusion}
\end{itemize}

\hypertarget{see-also}{\subsection{\texorpdfstring{\protect\hyperlink{see-also}{}See
also}{See also}}\label{see-also}}

\texttt{train.ruta\_autoencoder}

Other autoencoder variants: \texttt{autoencoder\_contractive},
\texttt{autoencoder\_denoising}, \texttt{autoencoder\_robust},
\texttt{autoencoder\_sparse}, \texttt{autoencoder\_variational}

\hypertarget{examples}{\subsection{\texorpdfstring{\protect\hyperlink{examples}{}Examples}{Examples}}\label{examples}}

\begin{minted}[frame=none]{R}
# Basic autoencoder with a network of [input]-256-36-256-[input] and
# no nonlinearities
autoencoder(c(256, 36), loss = "binary_crossentropy")
#> Autoencoder learner
#> ----------------------------------------
#> Type: basic 
#> 
#> Network structure:
#>  input
#>  dense(256 units) - linear
#>  dense(36 units) - linear
#>  dense(256 units) - linear
#>  dense - linear
#> 
#> Loss: binary_crossentropy 
#> ----------------------------------------
# Customizing the activation functions in the same network
network <-
  input() +
  dense(256, "relu") +
  dense(36, "tanh") +
  dense(256, "relu") +
  output("sigmoid")

learner <- autoencoder(
  network,
  loss = "binary_crossentropy"
)
\end{minted}

\section{Create a contractive
autoencoder}\label{create-a-contractive-autoencoder}

Source:
\href{https://github.com/fdavidcl/ruta/blob/master/R/autoencoder_contractive.R}{\texttt{R/autoencoder\_contractive.R}}

\texttt{autoencoder\_contractive.Rd}

A contractive autoencoder adds a penalty term to the loss function of a
basic autoencoder which attempts to induce a contraction of data in the
latent space.

\begin{minted}[frame=none]{R}
autoencoder_contractive(network, loss = "mean_squared_error",
  weight = 2e-04)
\end{minted}

\hypertarget{arguments}{\subsection{\texorpdfstring{\protect\hyperlink{arguments}{}Arguments}{Arguments}}\label{arguments}}

\begin{longtable}[c]{@{}>{\small}p{3cm}>{\raggedright}p{12.5cm}@{}}
\toprule
network & Layer construct of class
\texttt{``ruta\_network''}\tabularnewline
loss & Character string specifying the reconstruction error part of the
loss function\tabularnewline
weight & Weight assigned to the contractive loss\tabularnewline
\bottomrule
\end{longtable}

\hypertarget{value}{\subsection{\texorpdfstring{\protect\hyperlink{value}{}Value}{Value}}\label{value}}

A construct of class \texttt{``ruta\_autoencoder''}

\hypertarget{references}{\subsection{\texorpdfstring{\protect\hyperlink{references}{}References}{References}}\label{references}}

\begin{itemize}
\tightlist
\item
  \href{https://arxiv.org/abs/1801.01586}{A practical tutorial on
  autoencoders for nonlinear feature fusion}
\end{itemize}

\hypertarget{see-also}{\subsection{\texorpdfstring{\protect\hyperlink{see-also}{}See
also}{See also}}\label{see-also}}

Other autoencoder variants: \texttt{autoencoder\_denoising},
\texttt{autoencoder\_robust}, \texttt{autoencoder\_sparse},
\texttt{autoencoder\_variational}, \texttt{autoencoder}

\section{Create a denoising
autoencoder}\label{create-a-denoising-autoencoder}

Source:
\href{https://github.com/fdavidcl/ruta/blob/master/R/autoencoder_denoising.R}{\texttt{R/autoencoder\_denoising.R}}

\texttt{autoencoder\_denoising.Rd}

A denoising autoencoder trains with noisy data in order to create a
model able to reduce noise in reconstructions from input data

\begin{minted}[frame=none]{R}
autoencoder_denoising(network, loss = "mean_squared_error",
  noise_type = "zeros", ...)
\end{minted}

\hypertarget{arguments}{\subsection{\texorpdfstring{\protect\hyperlink{arguments}{}Arguments}{Arguments}}\label{arguments}}

\begin{longtable}[c]{@{}>{\small}p{3cm}>{\raggedright}p{12.5cm}@{}}
\toprule
\begin{minipage}[t]{0.07\columnwidth}\raggedright\strut
network
\strut\end{minipage} &
\begin{minipage}[t]{0.87\columnwidth}\raggedright\strut
Layer construct of class \texttt{``ruta\_network''}
\strut\end{minipage}\tabularnewline
\begin{minipage}[t]{0.07\columnwidth}\raggedright\strut
loss
\strut\end{minipage} &
\begin{minipage}[t]{0.87\columnwidth}\raggedright\strut
Loss function to be optimized
\strut\end{minipage}\tabularnewline
\begin{minipage}[t]{0.07\columnwidth}\raggedright\strut
noise\_type
\strut\end{minipage} &
\begin{minipage}[t]{0.87\columnwidth}\raggedright\strut
Type of data corruption which will be used to train the autoencoder, as
a character string. Available types:
\strut\end{minipage}\tabularnewline
\begin{minipage}[t]{0.07\columnwidth}\raggedright\strut
\strut\end{minipage} &
\begin{minipage}[t]{0.87\columnwidth}\raggedright\strut
- \texttt{``zeros''} Randomly set components to zero
(\texttt{noise\_zeros})
\strut\end{minipage}\tabularnewline
\begin{minipage}[t]{0.07\columnwidth}\raggedright\strut
\strut\end{minipage} &
\begin{minipage}[t]{0.87\columnwidth}\raggedright\strut
- \texttt{``ones''} Randomly set components to one (\texttt{noise\_ones})
\strut\end{minipage}\tabularnewline
\begin{minipage}[t]{0.07\columnwidth}\raggedright\strut
\strut\end{minipage} &
\begin{minipage}[t]{0.87\columnwidth}\raggedright\strut
- \texttt{``saltpepper''} Randomly set components to zero or one
(\texttt{noise\_saltpepper})
\strut\end{minipage}\tabularnewline
\begin{minipage}[t]{0.07\columnwidth}\raggedright\strut
\strut\end{minipage} &
\begin{minipage}[t]{0.87\columnwidth}\raggedright\strut
- \texttt{``gaussian''} Randomly offset each component of an input as
drawn from Gaussian distributions with the same variance (additive
Gaussian noise, \texttt{noise\_gaussian})
\strut\end{minipage}\tabularnewline
\begin{minipage}[t]{0.07\columnwidth}\raggedright\strut
\strut\end{minipage} &
\begin{minipage}[t]{0.87\columnwidth}\raggedright\strut
- \texttt{``cauchy''} Randomly offset each component of an input as drawn
from Cauchy distributions with the same scale (additive Cauchy noise,
\texttt{noise\_cauchy})
\strut\end{minipage}\tabularnewline
\begin{minipage}[t]{0.07\columnwidth}\raggedright\strut
\ldots{}
\strut\end{minipage} &
\begin{minipage}[t]{0.87\columnwidth}\raggedright\strut
Extra parameters to customize the noisy filter:
\strut\end{minipage}\tabularnewline
\begin{minipage}[t]{0.07\columnwidth}\raggedright\strut
\strut\end{minipage} &
\begin{minipage}[t]{0.87\columnwidth}\raggedright\strut
- \texttt{p} The probability that each instance in the input data which
will be altered by random noise (for \texttt{``zeros''}, \texttt{``ones''}
and \texttt{``saltpepper''})
\strut\end{minipage}\tabularnewline
\begin{minipage}[t]{0.07\columnwidth}\raggedright\strut
\strut\end{minipage} &
\begin{minipage}[t]{0.87\columnwidth}\raggedright\strut
- \texttt{var} or \texttt{sd} The variance or standard deviation of the
Gaussian distribution from which additive noise will be drawn (for
\texttt{``gaussian''}, only one of those parameters is necessary)
\strut\end{minipage}\tabularnewline
\begin{minipage}[t]{0.07\columnwidth}\raggedright\strut
\strut\end{minipage} &
\begin{minipage}[t]{0.87\columnwidth}\raggedright\strut
- \texttt{scale} For the Cauchy distribution
\strut\end{minipage}\tabularnewline
\bottomrule
\end{longtable}

\hypertarget{value}{\subsection{\texorpdfstring{\protect\hyperlink{value}{}Value}{Value}}\label{value}}

A construct of class \texttt{``ruta\_autoencoder''}

\hypertarget{references}{\subsection{\texorpdfstring{\protect\hyperlink{references}{}References}{References}}\label{references}}

\begin{itemize}
\tightlist
\item
  \href{https://dl.acm.org/citation.cfm?id=1390294}{Extracting and
  composing robust features with denoising autoencoders}
\end{itemize}

\hypertarget{see-also}{\subsection{\texorpdfstring{\protect\hyperlink{see-also}{}See
also}{See also}}\label{see-also}}

Other autoencoder variants: \texttt{autoencoder\_contractive},
\texttt{autoencoder\_robust}, \texttt{autoencoder\_sparse},
\texttt{autoencoder\_variational}, \texttt{autoencoder}

\section{Create a robust autoencoder}\label{create-a-robust-autoencoder}

Source:
\href{https://github.com/fdavidcl/ruta/blob/master/R/autoencoder_robust.R}{\texttt{R/autoencoder\_robust.R}}

\texttt{autoencoder\_robust.Rd}

A robust autoencoder uses a special objective function, correntropy, a
localized similarity measure which makes it less sensitive to noise in
data. Correntropy specifically measures the probability density that two
events are equal, and is less affected by outliers than the mean squared
error.

\begin{minted}[frame=none]{R}
autoencoder_robust(network, sigma = 0.2)
\end{minted}

\hypertarget{arguments}{\subsection{\texorpdfstring{\protect\hyperlink{arguments}{}Arguments}{Arguments}}\label{arguments}}

\begin{longtable}[c]{@{}>{\small}p{3cm}>{\raggedright}p{12.5cm}@{}}
\toprule
network & Layer construct of class
\texttt{``ruta\_network''}\tabularnewline
sigma & Sigma parameter in the kernel used for
correntropy\tabularnewline
\bottomrule
\end{longtable}

\hypertarget{value}{\subsection{\texorpdfstring{\protect\hyperlink{value}{}Value}{Value}}\label{value}}

A construct of class \texttt{``ruta\_autoencoder''}

\hypertarget{references}{\subsection{\texorpdfstring{\protect\hyperlink{references}{}References}{References}}\label{references}}

\begin{itemize}
\tightlist
\item
  \href{https://ieeexplore.ieee.org/abstract/document/6854900/}{Robust
  feature learning by stacked autoencoder with maximum correntropy
  criterion}
\end{itemize}

\hypertarget{see-also}{\subsection{\texorpdfstring{\protect\hyperlink{see-also}{}See
also}{See also}}\label{see-also}}

Other autoencoder variants: \texttt{autoencoder\_contractive},
\texttt{autoencoder\_denoising}, \texttt{autoencoder\_sparse},
\texttt{autoencoder\_variational}, \texttt{autoencoder}

\section{Sparse autoencoder}\label{sparse-autoencoder}

Source:
\href{https://github.com/fdavidcl/ruta/blob/master/R/autoencoder_sparse.R}{\texttt{R/autoencoder\_sparse.R}}

\texttt{autoencoder\_sparse.Rd}

Creates a representation of a sparse autoencoder.

\begin{minted}[frame=none]{R}
autoencoder_sparse(network, loss = "mean_squared_error",
  high_probability = 0.1, weight = 0.2)
\end{minted}

\hypertarget{arguments}{\subsection{\texorpdfstring{\protect\hyperlink{arguments}{}Arguments}{Arguments}}\label{arguments}}

\begin{longtable}[c]{@{}>{\small}p{3cm}>{\raggedright}p{12.5cm}@{}}
\toprule
network & Layer construct of class
\texttt{``ruta\_network''}\tabularnewline
loss & Character string specifying a loss function\tabularnewline
high\_probability & Expected probability of the high value of the
encoding layer. Set this to a value near zero in order to minimize
activations in that layer.\tabularnewline
weight & The weight of the sparsity regularization\tabularnewline
\bottomrule
\end{longtable}

\hypertarget{value}{\subsection{\texorpdfstring{\protect\hyperlink{value}{}Value}{Value}}\label{value}}

A construct of class \texttt{``ruta\_autoencoder''}

\hypertarget{references}{\subsection{\texorpdfstring{\protect\hyperlink{references}{}References}{References}}\label{references}}

\begin{itemize}
\item
  \href{http://papers.nips.cc/paper/3313-sparse-deep-belief-net-model-for-visual-area-v2}{Sparse
  deep belief net model for visual area V2}
\item
  Andrew Ng, Sparse Autoencoder.
  \href{https://web.stanford.edu/class/cs294a/sparseAutoencoder_2011new.pdf}{CS294A
  Lecture Notes}
\end{itemize}

\hypertarget{see-also}{\subsection{\texorpdfstring{\protect\hyperlink{see-also}{}See
also}{See also}}\label{see-also}}

\texttt{sparsity}, \texttt{make\_sparse}, \texttt{is\_sparse}

Other autoencoder variants: \texttt{autoencoder\_contractive},
\texttt{autoencoder\_denoising}, \texttt{autoencoder\_robust},
\texttt{autoencoder\_variational}, \texttt{autoencoder}

\section{Build a variational
autoencoder}\label{build-a-variational-autoencoder}

Source:
\href{https://github.com/fdavidcl/ruta/blob/master/R/autoencoder_variational.R}{\texttt{R/autoencoder\_variational.R}}

\texttt{autoencoder\_variational.Rd}

A variational autoencoder assumes that a latent, unobserved random
variable produces the observed data and attempts to approximate its
distribution. This function constructs a wrapper for a variational
autoencoder using a Gaussian distribution as the prior of the latent
space.

\begin{minted}[frame=none]{R}
autoencoder_variational(network, loss = "binary_crossentropy",
  auto_transform_network = TRUE)
\end{minted}

\hypertarget{arguments}{\subsection{\texorpdfstring{\protect\hyperlink{arguments}{}Arguments}{Arguments}}\label{arguments}}

\begin{longtable}[c]{@{}>{\small}p{3cm}>{\raggedright}p{12.5cm}@{}}
\toprule
network & Network architecture as a \texttt{``ruta\_network''} object (or
coercible)\tabularnewline
loss & Reconstruction error to be combined with KL divergence in order
to compute the variational loss\tabularnewline
auto\_transform\_network & Boolean: convert the encoding layer into a
variational block if none is found?\tabularnewline
\bottomrule
\end{longtable}

\hypertarget{value}{\subsection{\texorpdfstring{\protect\hyperlink{value}{}Value}{Value}}\label{value}}

A construct of class \texttt{``ruta\_autoencoder''}

\hypertarget{references}{\subsection{\texorpdfstring{\protect\hyperlink{references}{}References}{References}}\label{references}}

\begin{itemize}
\item
  \href{https://arxiv.org/abs/1312.6114}{Auto-Encoding Variational
  Bayes}
\item
  \href{http://blog.fastforwardlabs.com/2016/08/22/under-the-hood-of-the-variational-autoencoder-in.html}{Under
  the Hood of the Variational Autoencoder (in Prose and Code)}
\item
  \href{https://keras.rstudio.com/articles/examples/variational_autoencoder.html}{Keras
  example: Variational autoencoder}
\end{itemize}

\hypertarget{see-also}{\subsection{\texorpdfstring{\protect\hyperlink{see-also}{}See
also}{See also}}\label{see-also}}

Other autoencoder variants: \texttt{autoencoder\_contractive},
\texttt{autoencoder\_denoising}, \texttt{autoencoder\_robust},
\texttt{autoencoder\_sparse}, \texttt{autoencoder}

\hypertarget{examples}{\subsection{\texorpdfstring{\protect\hyperlink{examples}{}Examples}{Examples}}\label{examples}}

\begin{minted}[frame=none]{R}
network <-
  input() +
  dense(256, "elu") +
  variational_block(3) +
  dense(256, "elu") +
  output("sigmoid")

learner <- autoencoder_variational(network, loss = "binary_crossentropy")
\end{minted}

\section{Contractive loss}\label{contractive-loss}

Source:
\href{https://github.com/fdavidcl/ruta/blob/master/R/autoencoder_contractive.R}{\texttt{R/autoencoder\_contractive.R}}

\texttt{contraction.Rd}

This is a wrapper for a loss which induces a contraction in the latent
space.

\begin{minted}[frame=none]{R}
contraction(reconstruction_loss = "mean_squared_error", weight = 2e-04)
\end{minted}

\hypertarget{arguments}{\subsection{\texorpdfstring{\protect\hyperlink{arguments}{}Arguments}{Arguments}}\label{arguments}}

\begin{longtable}[c]{@{}>{\small}p{3cm}>{\raggedright}p{12.5cm}@{}}
\toprule
reconstruction\_loss & Original reconstruction error to be combined with
the contractive loss (e.g.
\texttt{``binary\_crossentropy''})\tabularnewline
weight & Weight assigned to the contractive loss\tabularnewline
\bottomrule
\end{longtable}

\hypertarget{value}{\subsection{\texorpdfstring{\protect\hyperlink{value}{}Value}{Value}}\label{value}}

A loss object which can be converted into a Keras loss

\hypertarget{see-also}{\subsection{\texorpdfstring{\protect\hyperlink{see-also}{}See
also}{See also}}\label{see-also}}

\texttt{autoencoder\_contractive}

Other loss functions: \texttt{correntropy}, \texttt{loss\_variational}

\section{Correntropy loss}\label{correntropy-loss}

Source:
\href{https://github.com/fdavidcl/ruta/blob/master/R/autoencoder_robust.R}{\texttt{R/autoencoder\_robust.R}}

\texttt{correntropy.Rd}

A wrapper for the correntropy loss function

\begin{minted}[frame=none]{R}
correntropy(sigma = 0.2)
\end{minted}

\hypertarget{arguments}{\subsection{\texorpdfstring{\protect\hyperlink{arguments}{}Arguments}{Arguments}}\label{arguments}}

\begin{longtable}[c]{@{}>{\small}p{3cm}>{\raggedright}p{12.5cm}@{}}
\toprule
sigma & Sigma parameter in the kernel\tabularnewline
\bottomrule
\end{longtable}

\hypertarget{value}{\subsection{\texorpdfstring{\protect\hyperlink{value}{}Value}{Value}}\label{value}}

A \texttt{``ruta\_loss''} object

\hypertarget{see-also}{\subsection{\texorpdfstring{\protect\hyperlink{see-also}{}See
also}{See also}}\label{see-also}}

\texttt{autoencoder\_robust}

Other loss functions: \texttt{contraction}, \texttt{loss\_variational}

\section{Retrieve decoding of encoded
data}\label{retrieve-decoding-of-encoded-data}

Source:
\href{https://github.com/fdavidcl/ruta/blob/master/R/autoencoder.R}{\texttt{R/autoencoder.R}}

\texttt{decode.Rd}

Extracts the decodification calculated by a trained autoencoder for the
specified data.

\begin{minted}[frame=none]{R}
decode(learner, data)
\end{minted}

\hypertarget{arguments}{\subsection{\texorpdfstring{\protect\hyperlink{arguments}{}Arguments}{Arguments}}\label{arguments}}

\begin{longtable}[c]{@{}>{\small}p{3cm}>{\raggedright}p{12.5cm}@{}}
\toprule
learner & Trained autoencoder model\tabularnewline
data & data.frame to be decoded\tabularnewline
\bottomrule
\end{longtable}

\hypertarget{value}{\subsection{\texorpdfstring{\protect\hyperlink{value}{}Value}{Value}}\label{value}}

Matrix containing the decodifications

\hypertarget{see-also}{\subsection{\texorpdfstring{\protect\hyperlink{see-also}{}See
also}{See also}}\label{see-also}}

\texttt{encode}, \texttt{reconstruct}

\section{Create a fully-connected neural
layer}\label{create-a-fully-connected-neural-layer}

Source:
\href{https://github.com/fdavidcl/ruta/blob/master/R/layers.R}{\texttt{R/layers.R}}

\texttt{dense.Rd}

Wrapper for a dense/fully-connected layer.

\begin{minted}[frame=none]{R}
dense(units, activation = "linear")
\end{minted}

\hypertarget{arguments}{\subsection{\texorpdfstring{\protect\hyperlink{arguments}{}Arguments}{Arguments}}\label{arguments}}

\begin{longtable}[c]{@{}>{\small}p{3cm}>{\raggedright}p{12.5cm}@{}}
\toprule
units & Number of units\tabularnewline
activation & Optional, string indicating activation function (linear by
default)\tabularnewline
\bottomrule
\end{longtable}

\hypertarget{value}{\subsection{\texorpdfstring{\protect\hyperlink{value}{}Value}{Value}}\label{value}}

A construct with class \texttt{``ruta\_network''}

\hypertarget{see-also}{\subsection{\texorpdfstring{\protect\hyperlink{see-also}{}See
also}{See also}}\label{see-also}}

Other neural layers: \texttt{dropout}, \texttt{input},
\texttt{layer\_keras}, \texttt{output}, \texttt{variational\_block}

\hypertarget{examples}{\subsection{\texorpdfstring{\protect\hyperlink{examples}{}Examples}{Examples}}\label{examples}}

\begin{minted}[frame=none]{R}
dense(30, "tanh")
#> Network structure:
#>  dense(30 units) - tanh
\end{minted}

\section{Dropout layer}\label{dropout-layer}

Source:
\href{https://github.com/fdavidcl/ruta/blob/master/R/layers.R}{\texttt{R/layers.R}}

\texttt{dropout.Rd}

Randomly sets a fraction \texttt{rate} of input units to 0 at each
update during training time, which helps prevent overfitting.

\begin{minted}[frame=none]{R}
dropout(rate = 0.5)
\end{minted}

\hypertarget{arguments}{\subsection{\texorpdfstring{\protect\hyperlink{arguments}{}Arguments}{Arguments}}\label{arguments}}

\begin{longtable}[c]{@{}>{\small}p{3cm}>{\raggedright}p{12.5cm}@{}}
\toprule
rate & The fraction of affected units\tabularnewline
\bottomrule
\end{longtable}

\hypertarget{value}{\subsection{\texorpdfstring{\protect\hyperlink{value}{}Value}{Value}}\label{value}}

A construct of class \texttt{``ruta\_network''}

\hypertarget{see-also}{\subsection{\texorpdfstring{\protect\hyperlink{see-also}{}See
also}{See also}}\label{see-also}}

Other neural layers: \texttt{dense}, \texttt{input},
\texttt{layer\_keras}, \texttt{output}, \texttt{variational\_block}

\section{Retrieve encoding of data}\label{retrieve-encoding-of-data}

Source:
\href{https://github.com/fdavidcl/ruta/blob/master/R/autoencoder.R}{\texttt{R/autoencoder.R}}

\texttt{encode.Rd}

Extracts the encoding calculated by a trained autoencoder for the
specified data.

\begin{minted}[frame=none]{R}
encode(learner, data)
\end{minted}

\hypertarget{arguments}{\subsection{\texorpdfstring{\protect\hyperlink{arguments}{}Arguments}{Arguments}}\label{arguments}}

\begin{longtable}[c]{@{}>{\small}p{3cm}>{\raggedright}p{12.5cm}@{}}
\toprule
learner & Trained autoencoder model\tabularnewline
data & data.frame to be encoded\tabularnewline
\bottomrule
\end{longtable}

\hypertarget{value}{\subsection{\texorpdfstring{\protect\hyperlink{value}{}Value}{Value}}\label{value}}

Matrix containing the encodings

\hypertarget{see-also}{\subsection{\texorpdfstring{\protect\hyperlink{see-also}{}See
also}{See also}}\label{see-also}}

\texttt{decode}, \texttt{reconstruct}

\section{Get the index of the
encoding}\label{get-the-index-of-the-encoding}

Source:
\href{https://github.com/fdavidcl/ruta/blob/master/R/network.R}{\texttt{R/network.R}}

\texttt{encoding\_index.Rd}

Calculates the index of the middle layer of an encoder-decoder network.

\begin{minted}[frame=none]{R}
encoding_index(net)
\end{minted}

\hypertarget{arguments}{\subsection{\texorpdfstring{\protect\hyperlink{arguments}{}Arguments}{Arguments}}\label{arguments}}

\begin{longtable}[c]{@{}>{\small}p{3cm}>{\raggedright}p{12.5cm}@{}}
\toprule
net & A network of class \texttt{``ruta\_network''}\tabularnewline
\bottomrule
\end{longtable}

\hypertarget{value}{\subsection{\texorpdfstring{\protect\hyperlink{value}{}Value}{Value}}\label{value}}

Index of the middle layer

\section{Evaluation metrics}\label{evaluation-metrics}

Source:
\href{https://github.com/fdavidcl/ruta/blob/master/R/evaluate.R}{\texttt{R/evaluate.R}}

\texttt{evaluate.Rd}

Performance evaluation metrics for autoencoders

\begin{minted}[frame=none]{R}
evaluate_mean_squared_error(learner, data)

evaluate_mean_absolute_error(learner, data)

evaluate_binary_crossentropy(learner, data)

evaluate_binary_accuracy(learner, data)

evaluate_kullback_leibler_divergence(learner, data)
\end{minted}

\hypertarget{arguments}{\subsection{\texorpdfstring{\protect\hyperlink{arguments}{}Arguments}{Arguments}}\label{arguments}}

\begin{longtable}[c]{@{}>{\small}p{3cm}>{\raggedright}p{12.5cm}@{}}
\toprule
learner & A trained learner object\tabularnewline
data & Test data for evaluation\tabularnewline
\bottomrule
\end{longtable}

\hypertarget{value}{\subsection{\texorpdfstring{\protect\hyperlink{value}{}Value}{Value}}\label{value}}

A named list with the autoencoder training loss and evaluation metric
for the given data

\hypertarget{see-also}{\subsection{\texorpdfstring{\protect\hyperlink{see-also}{}See
also}{See also}}\label{see-also}}

\texttt{evaluation\_metric}

\hypertarget{examples}{\subsection{\texorpdfstring{\protect\hyperlink{examples}{}Examples}{Examples}}\label{examples}}

\begin{minted}[frame=none]{R}
library(purrr)

x <- as.matrix(sample(iris[, 1:4]))
x_train <- x[1:100, ]
x_test <- x[101:150, ]autoencoder(2) %>%
  train(x_train) %>%
  evaluate_mean_squared_error(x_test)
#> $loss
#> [1] 23.26905
#> 
#> $mean_squared_error
#> [1] 23.26905
#> 
\end{minted}

\section{Custom evaluation metrics}\label{custom-evaluation-metrics}

Source:
\href{https://github.com/fdavidcl/ruta/blob/master/R/evaluate.R}{\texttt{R/evaluate.R}}

\texttt{evaluation\_metric.Rd}

Create a different evaluation metric from a valid Keras metric

\begin{minted}[frame=none]{R}
evaluation_metric(evaluate_f)
\end{minted}

\hypertarget{arguments}{\subsection{\texorpdfstring{\protect\hyperlink{arguments}{}Arguments}{Arguments}}\label{arguments}}

\begin{longtable}[c]{@{}>{\small}p{3cm}>{\raggedright}p{12.5cm}@{}}
\toprule
evaluate\_f & Must be either a metric function defined by Keras (e.g.
\texttt{keras::metric\_binary\_crossentropy}) or a valid function for
Keras to create a performance metric (see
\texttt{metric\_binary\_accuracy} for details)\tabularnewline
\bottomrule
\end{longtable}

\hypertarget{value}{\subsection{\texorpdfstring{\protect\hyperlink{value}{}Value}{Value}}\label{value}}

A function which can be called with parameters \texttt{learner} and
\texttt{data} just like the ones defined in \texttt{evaluate}.

\hypertarget{see-also}{\subsection{\texorpdfstring{\protect\hyperlink{see-also}{}See
also}{See also}}\label{see-also}}

\texttt{evaluate}

\section{Generate samples from a generative
model}\label{generate-samples-from-a-generative-model}

Source:
\href{https://github.com/fdavidcl/ruta/blob/master/R/autoencoder_variational.R}{\texttt{R/autoencoder\_variational.R}},
\href{https://github.com/fdavidcl/ruta/blob/master/R/generics.R}{\texttt{R/generics.R}}

\texttt{generate.Rd}

Generate samples from a generative model

\begin{minted}[frame=none]{R}
# S3 method for ruta_autoencoder_variational
generate(learner, dimensions = c(1, 2),
  from = 0.05, to = 0.95, side = 10, fixed_values = 0.5, ...)

generate(learner, ...)
\end{minted}

\hypertarget{arguments}{\subsection{\texorpdfstring{\protect\hyperlink{arguments}{}Arguments}{Arguments}}\label{arguments}}

\begin{longtable}[c]{@{}>{\small}p{3cm}>{\raggedright}p{12.5cm}@{}}
\toprule
learner & Trained learner object\tabularnewline
dimensions & Indices of the dimensions over which the model will be
sampled\tabularnewline
from & Lower limit on the values which will be passed to the inverse CDF
of the prior\tabularnewline
to & Upper limit on the values which will be passed to the inverse CDF
of the prior\tabularnewline
side & Number of steps to take in each traversed
dimension\tabularnewline
fixed\_values & Value used as parameter for the inverse CDF of all
non-traversed dimensions\tabularnewline
\ldots{} & Unused\tabularnewline
\bottomrule
\end{longtable}

\hypertarget{see-also}{\subsection{\texorpdfstring{\protect\hyperlink{see-also}{}See
also}{See also}}\label{see-also}}

\texttt{autoencoder\_variational}

\section{Create an input layer}\label{create-an-input-layer}

Source:
\href{https://github.com/fdavidcl/ruta/blob/master/R/layers.R}{\texttt{R/layers.R}}

\texttt{input.Rd}

This layer acts as a placeholder for input data. The number of units is
not needed as it is deduced from the data during training.

\begin{minted}[frame=none]{R}
input()
\end{minted}

\hypertarget{value}{\subsection{\texorpdfstring{\protect\hyperlink{value}{}Value}{Value}}\label{value}}

A construct with class \texttt{``ruta\_network''}

\hypertarget{see-also}{\subsection{\texorpdfstring{\protect\hyperlink{see-also}{}See
also}{See also}}\label{see-also}}

Other neural layers: \texttt{dense}, \texttt{dropout},
\texttt{layer\_keras}, \texttt{output}, \texttt{variational\_block}

\section{Detect whether an autoencoder is
contractive}\label{detect-whether-an-autoencoder-is-contractive}

Source:
\href{https://github.com/fdavidcl/ruta/blob/master/R/autoencoder_contractive.R}{\texttt{R/autoencoder\_contractive.R}}

\texttt{is\_contractive.Rd}

Detect whether an autoencoder is contractive

\begin{minted}[frame=none]{R}
is_contractive(learner)
\end{minted}

\hypertarget{arguments}{\subsection{\texorpdfstring{\protect\hyperlink{arguments}{}Arguments}{Arguments}}\label{arguments}}

\begin{longtable}[c]{@{}>{\small}p{3cm}>{\raggedright}p{12.5cm}@{}}
\toprule
learner & A \texttt{``ruta\_autoencoder''} object\tabularnewline
\bottomrule
\end{longtable}

\hypertarget{value}{\subsection{\texorpdfstring{\protect\hyperlink{value}{}Value}{Value}}\label{value}}

Logical value indicating if a contractive loss was found

\hypertarget{see-also}{\subsection{\texorpdfstring{\protect\hyperlink{see-also}{}See
also}{See also}}\label{see-also}}

\texttt{contraction}, \texttt{autoencoder\_contractive},
\texttt{make\_contractive}

\section{Detect whether an autoencoder is
denoising}\label{detect-whether-an-autoencoder-is-denoising}

Source:
\href{https://github.com/fdavidcl/ruta/blob/master/R/autoencoder_denoising.R}{\texttt{R/autoencoder\_denoising.R}}

\texttt{is\_denoising.Rd}

Detect whether an autoencoder is denoising

\begin{minted}[frame=none]{R}
is_denoising(learner)
\end{minted}

\hypertarget{arguments}{\subsection{\texorpdfstring{\protect\hyperlink{arguments}{}Arguments}{Arguments}}\label{arguments}}

\begin{longtable}[c]{@{}>{\small}p{3cm}>{\raggedright}p{12.5cm}@{}}
\toprule
learner & A \texttt{``ruta\_autoencoder''} object\tabularnewline
\bottomrule
\end{longtable}

\hypertarget{value}{\subsection{\texorpdfstring{\protect\hyperlink{value}{}Value}{Value}}\label{value}}

Logical value indicating if a noise generator was found

\hypertarget{see-also}{\subsection{\texorpdfstring{\protect\hyperlink{see-also}{}See
also}{See also}}\label{see-also}}

\texttt{noise}, \texttt{autoencoder\_denoising},
\texttt{make\_denoising}

\section{Detect whether an autoencoder is
robust}\label{detect-whether-an-autoencoder-is-robust}

Source:
\href{https://github.com/fdavidcl/ruta/blob/master/R/autoencoder_robust.R}{\texttt{R/autoencoder\_robust.R}}

\texttt{is\_robust.Rd}

Detect whether an autoencoder is robust

\begin{minted}[frame=none]{R}
is_robust(learner)
\end{minted}

\hypertarget{arguments}{\subsection{\texorpdfstring{\protect\hyperlink{arguments}{}Arguments}{Arguments}}\label{arguments}}

\begin{longtable}[c]{@{}>{\small}p{3cm}>{\raggedright}p{12.5cm}@{}}
\toprule
learner & A \texttt{``ruta\_autoencoder''} object\tabularnewline
\bottomrule
\end{longtable}

\hypertarget{value}{\subsection{\texorpdfstring{\protect\hyperlink{value}{}Value}{Value}}\label{value}}

Logical value indicating if a correntropy loss was found

\hypertarget{see-also}{\subsection{\texorpdfstring{\protect\hyperlink{see-also}{}See
also}{See also}}\label{see-also}}

\texttt{correntropy}, \texttt{autoencoder\_robust},
\texttt{make\_robust}

\section{Detect whether an autoencoder is
sparse}\label{detect-whether-an-autoencoder-is-sparse}

Source:
\href{https://github.com/fdavidcl/ruta/blob/master/R/autoencoder_sparse.R}{\texttt{R/autoencoder\_sparse.R}}

\texttt{is\_sparse.Rd}

Detect whether an autoencoder is sparse

\begin{minted}[frame=none]{R}
is_sparse(learner)
\end{minted}

\hypertarget{arguments}{\subsection{\texorpdfstring{\protect\hyperlink{arguments}{}Arguments}{Arguments}}\label{arguments}}

\begin{longtable}[c]{@{}>{\small}p{3cm}>{\raggedright}p{12.5cm}@{}}
\toprule
learner & A \texttt{``ruta\_autoencoder''} object\tabularnewline
\bottomrule
\end{longtable}

\hypertarget{value}{\subsection{\texorpdfstring{\protect\hyperlink{value}{}Value}{Value}}\label{value}}

Logical value indicating if a sparsity regularization in the encoding
layer was found

\hypertarget{see-also}{\subsection{\texorpdfstring{\protect\hyperlink{see-also}{}See
also}{See also}}\label{see-also}}

\texttt{sparsity}, \texttt{autoencoder\_sparse}, \texttt{make\_sparse}

\section{Detect trained models}\label{detect-trained-models}

Source:
\href{https://github.com/fdavidcl/ruta/blob/master/R/autoencoder.R}{\texttt{R/autoencoder.R}}

\texttt{is\_trained.Rd}

Inspects a learner and figures out whether it has been trained

\begin{minted}[frame=none]{R}
is_trained(learner)
\end{minted}

\hypertarget{arguments}{\subsection{\texorpdfstring{\protect\hyperlink{arguments}{}Arguments}{Arguments}}\label{arguments}}

\begin{longtable}[c]{@{}>{\small}p{3cm}>{\raggedright}p{12.5cm}@{}}
\toprule
learner & Learner object\tabularnewline
\bottomrule
\end{longtable}

\hypertarget{value}{\subsection{\texorpdfstring{\protect\hyperlink{value}{}Value}{Value}}\label{value}}

A boolean

\hypertarget{see-also}{\subsection{\texorpdfstring{\protect\hyperlink{see-also}{}See
also}{See also}}\label{see-also}}

\texttt{train}

\section{Detect whether an autoencoder is
variational}\label{detect-whether-an-autoencoder-is-variational}

Source:
\href{https://github.com/fdavidcl/ruta/blob/master/R/autoencoder_variational.R}{\texttt{R/autoencoder\_variational.R}}

\texttt{is\_variational.Rd}

Detect whether an autoencoder is variational

\begin{minted}[frame=none]{R}
is_variational(learner)
\end{minted}

\hypertarget{arguments}{\subsection{\texorpdfstring{\protect\hyperlink{arguments}{}Arguments}{Arguments}}\label{arguments}}

\begin{longtable}[c]{@{}>{\small}p{3cm}>{\raggedright}p{12.5cm}@{}}
\toprule
learner & A \texttt{``ruta\_autoencoder''} object\tabularnewline
\bottomrule
\end{longtable}

\hypertarget{value}{\subsection{\texorpdfstring{\protect\hyperlink{value}{}Value}{Value}}\label{value}}

Logical value indicating if a variational loss was found

\hypertarget{see-also}{\subsection{\texorpdfstring{\protect\hyperlink{see-also}{}See
also}{See also}}\label{see-also}}

\texttt{autoencoder\_variational}

\section{Add layers to a network/Join
networks}\label{add-layers-to-a-networkjoin-networks}

Source:
\href{https://github.com/fdavidcl/ruta/blob/master/R/network.R}{\texttt{R/network.R}}

\texttt{join-networks.Rd}

Add layers to a network/Join networks

\begin{minted}[frame=none]{R}
# S3 method for ruta_network
+(e1, e2)

# S3 method for ruta_network
c(...)
\end{minted}

\hypertarget{arguments}{\subsection{\texorpdfstring{\protect\hyperlink{arguments}{}Arguments}{Arguments}}\label{arguments}}

\begin{longtable}[c]{@{}>{\small}p{3cm}>{\raggedright}p{12.5cm}@{}}
\toprule
e1 & First network\tabularnewline
e2 & Second network\tabularnewline
\ldots{} & networks or layers to be concatenated\tabularnewline
\bottomrule
\end{longtable}

\hypertarget{value}{\subsection{\texorpdfstring{\protect\hyperlink{value}{}Value}{Value}}\label{value}}

Network combination

\hypertarget{examples}{\subsection{\texorpdfstring{\protect\hyperlink{examples}{}Examples}{Examples}}\label{examples}}

\begin{minted}[frame=none]{R}
network <- input() + dense(30) + output("sigmoid")
another <- c(input(), dense(30), dense(3), dense(30), output())
\end{minted}

\section{Custom layer from Keras}\label{custom-layer-from-keras}

Source:
\href{https://github.com/fdavidcl/ruta/blob/master/R/layers.R}{\texttt{R/layers.R}}

\texttt{layer\_keras.Rd}

Gets any layer available in Keras with the specified parameters

\begin{minted}[frame=none]{R}
layer_keras(name, ...)
\end{minted}

\hypertarget{arguments}{\subsection{\texorpdfstring{\protect\hyperlink{arguments}{}Arguments}{Arguments}}\label{arguments}}

\begin{longtable}[c]{@{}>{\small}p{3cm}>{\raggedright}p{12.5cm}@{}}
\toprule
name & The name of the layer, e.g. \texttt{``activity\_regularization''}
for a \texttt{keras::layer\_activity\_regularization}
object\tabularnewline
\ldots{} & Named parameters for the Keras layer
constructor\tabularnewline
\bottomrule
\end{longtable}

\hypertarget{value}{\subsection{\texorpdfstring{\protect\hyperlink{value}{}Value}{Value}}\label{value}}

A wrapper for the specified layer, which can be combined with other Ruta
layers

\hypertarget{see-also}{\subsection{\texorpdfstring{\protect\hyperlink{see-also}{}See
also}{See also}}\label{see-also}}

Other neural layers: \texttt{dense}, \texttt{dropout}, \texttt{input},
\texttt{output}, \texttt{variational\_block}

\section{Variational loss}\label{variational-loss}

Source:
\href{https://github.com/fdavidcl/ruta/blob/master/R/autoencoder_variational.R}{\texttt{R/autoencoder\_variational.R}}

\texttt{loss\_variational.Rd}

Specifies an evaluation function adapted to the variational autoencoder.
It combines a base reconstruction error and the Kullback-Leibler
divergence between the learned distribution and the true latent
posterior.

\begin{minted}[frame=none]{R}
loss_variational(reconstruction_loss)
\end{minted}

\hypertarget{arguments}{\subsection{\texorpdfstring{\protect\hyperlink{arguments}{}Arguments}{Arguments}}\label{arguments}}

\begin{longtable}[c]{@{}>{\small}p{3cm}>{\raggedright}p{12.5cm}@{}}
\toprule
reconstruction\_loss & Another loss to be used as reconstruction error
(e.g. ``binary\_crossentropy'')\tabularnewline
\bottomrule
\end{longtable}

\hypertarget{value}{\subsection{\texorpdfstring{\protect\hyperlink{value}{}Value}{Value}}\label{value}}

A \texttt{``ruta\_loss''} object

\hypertarget{references}{\subsection{\texorpdfstring{\protect\hyperlink{references}{}References}{References}}\label{references}}

\begin{itemize}
\item
  \href{https://arxiv.org/abs/1312.6114}{Auto-Encoding Variational
  Bayes}
\item
  \href{http://blog.fastforwardlabs.com/2016/08/22/under-the-hood-of-the-variational-autoencoder-in.html}{Under
  the Hood of the Variational Autoencoder (in Prose and Code)}
\item
  \href{https://keras.rstudio.com/articles/examples/variational_autoencoder.html}{Keras
  example: Variational autoencoder}
\end{itemize}

\hypertarget{see-also}{\subsection{\texorpdfstring{\protect\hyperlink{see-also}{}See
also}{See also}}\label{see-also}}

\texttt{autoencoder\_variational}

Other loss functions: \texttt{contraction}, \texttt{correntropy}

\section{Add contractive behavior to any
autoencoder}\label{add-contractive-behavior-to-any-autoencoder}

Source:
\href{https://github.com/fdavidcl/ruta/blob/master/R/autoencoder_contractive.R}{\texttt{R/autoencoder\_contractive.R}}

\texttt{make\_contractive.Rd}

Converts an autoencoder into a contractive one by assigning a
contractive loss to it

\begin{minted}[frame=none]{R}
make_contractive(learner, weight = 2e-04)
\end{minted}

\hypertarget{arguments}{\subsection{\texorpdfstring{\protect\hyperlink{arguments}{}Arguments}{Arguments}}\label{arguments}}

\begin{longtable}[c]{@{}>{\small}p{3cm}>{\raggedright}p{12.5cm}@{}}
\toprule
learner & The \texttt{``ruta\_autoencoder''} object\tabularnewline
weight & Weight assigned to the contractive loss\tabularnewline
\bottomrule
\end{longtable}

\hypertarget{value}{\subsection{\texorpdfstring{\protect\hyperlink{value}{}Value}{Value}}\label{value}}

An autoencoder object which contains the contractive loss

\hypertarget{see-also}{\subsection{\texorpdfstring{\protect\hyperlink{see-also}{}See
also}{See also}}\label{see-also}}

\texttt{autoencoder\_contractive}

\section{Add denoising behavior to any
autoencoder}\label{add-denoising-behavior-to-any-autoencoder}

Source:
\href{https://github.com/fdavidcl/ruta/blob/master/R/autoencoder_denoising.R}{\texttt{R/autoencoder\_denoising.R}}

\texttt{make\_denoising.Rd}

Converts an autoencoder into a denoising one by adding a filter for the
input data

\begin{minted}[frame=none]{R}
make_denoising(learner, noise_type = "zeros", ...)
\end{minted}

\hypertarget{arguments}{\subsection{\texorpdfstring{\protect\hyperlink{arguments}{}Arguments}{Arguments}}\label{arguments}}

\begin{longtable}[c]{@{}>{\small}p{3cm}>{\raggedright}p{12.5cm}@{}}
\toprule
learner & The \texttt{``ruta\_autoencoder''} object\tabularnewline
noise\_type & Type of data corruption which will be used to train the
autoencoder, as a character string. See \texttt{autoencoder\_denoising}
for details\tabularnewline
\ldots{} & Extra parameters to customize the noisy filter. See
\texttt{autoencoder\_denoising} for details\tabularnewline
\bottomrule
\end{longtable}

\hypertarget{value}{\subsection{\texorpdfstring{\protect\hyperlink{value}{}Value}{Value}}\label{value}}

An autoencoder object which contains the noisy filter

\hypertarget{see-also}{\subsection{\texorpdfstring{\protect\hyperlink{see-also}{}See
also}{See also}}\label{see-also}}

\texttt{autoencoder\_denoising}

\section{Add robust behavior to any
autoencoder}\label{add-robust-behavior-to-any-autoencoder}

Source:
\href{https://github.com/fdavidcl/ruta/blob/master/R/autoencoder_robust.R}{\texttt{R/autoencoder\_robust.R}}

\texttt{make\_robust.Rd}

Converts an autoencoder into a robust one by assigning a correntropy
loss to it. Notice that this will replace the previous loss function

\begin{minted}[frame=none]{R}
make_robust(learner, sigma = 0.2)
\end{minted}

\hypertarget{arguments}{\subsection{\texorpdfstring{\protect\hyperlink{arguments}{}Arguments}{Arguments}}\label{arguments}}

\begin{longtable}[c]{@{}>{\small}p{3cm}>{\raggedright}p{12.5cm}@{}}
\toprule
learner & The \texttt{``ruta\_autoencoder''} object\tabularnewline
sigma & Sigma parameter in the kernel used for
correntropy\tabularnewline
\bottomrule
\end{longtable}

\hypertarget{value}{\subsection{\texorpdfstring{\protect\hyperlink{value}{}Value}{Value}}\label{value}}

An autoencoder object which contains the correntropy loss

\hypertarget{see-also}{\subsection{\texorpdfstring{\protect\hyperlink{see-also}{}See
also}{See also}}\label{see-also}}

\texttt{autoencoder\_robust}

\section{Add sparsity regularization to an
autoencoder}\label{add-sparsity-regularization-to-an-autoencoder}

Source:
\href{https://github.com/fdavidcl/ruta/blob/master/R/autoencoder_sparse.R}{\texttt{R/autoencoder\_sparse.R}}

\texttt{make\_sparse.Rd}

Add sparsity regularization to an autoencoder

\begin{minted}[frame=none]{R}
make_sparse(learner, high_probability = 0.1, weight = 0.2)
\end{minted}

\hypertarget{arguments}{\subsection{\texorpdfstring{\protect\hyperlink{arguments}{}Arguments}{Arguments}}\label{arguments}}

\begin{longtable}[c]{@{}>{\small}p{3cm}>{\raggedright}p{12.5cm}@{}}
\toprule
learner & A \texttt{``ruta\_autoencoder''} object\tabularnewline
high\_probability & Expected probability of the high value of the
encoding layer. Set this to a value near zero in order to minimize
activations in that layer.\tabularnewline
weight & The weight of the sparsity regularization\tabularnewline
\bottomrule
\end{longtable}

\hypertarget{value}{\subsection{\texorpdfstring{\protect\hyperlink{value}{}Value}{Value}}\label{value}}

The same autoencoder with the sparsity regularization applied

\hypertarget{see-also}{\subsection{\texorpdfstring{\protect\hyperlink{see-also}{}See
also}{See also}}\label{see-also}}

\texttt{sparsity}, \texttt{autoencoder\_sparse}, \texttt{is\_sparse}

\section{Create an autoencoder
learner}\label{create-an-autoencoder-learner-1}

Source:
\href{https://github.com/fdavidcl/ruta/blob/master/R/autoencoder.R}{\texttt{R/autoencoder.R}}

\texttt{new\_autoencoder.Rd}

Internal function to create autoencoder objects. Instead, consider using
\texttt{autoencoder}.

\begin{minted}[frame=none]{R}
new_autoencoder(network, loss, extra_class = NULL)
\end{minted}

\hypertarget{arguments}{\subsection{\texorpdfstring{\protect\hyperlink{arguments}{}Arguments}{Arguments}}\label{arguments}}

\begin{longtable}[c]{@{}>{\small}p{3cm}>{\raggedright}p{12.5cm}@{}}
\toprule
network & Layer construct of class \texttt{``ruta\_network''} or
coercible\tabularnewline
loss & A \texttt{``ruta\_loss''} object or a character string specifying a
loss function\tabularnewline
extra\_class & Vector of classes in case this autoencoder needs to
support custom methods (for \texttt{to\_keras}, \texttt{train},
\texttt{generate} or others)\tabularnewline
\bottomrule
\end{longtable}

\hypertarget{value}{\subsection{\texorpdfstring{\protect\hyperlink{value}{}Value}{Value}}\label{value}}

A construct of class \texttt{``ruta\_autoencoder''}

\section{Layer wrapper constructor}\label{layer-wrapper-constructor}

Source:
\href{https://github.com/fdavidcl/ruta/blob/master/R/layers.R}{\texttt{R/layers.R}}

\texttt{new\_layer.Rd}

Constructor function for layers. You shouldn't generally need to use
this. Instead, consider using individual functions such as
\texttt{dense}.

\begin{minted}[frame=none]{R}
new_layer(cl, ...)
\end{minted}

\hypertarget{arguments}{\subsection{\texorpdfstring{\protect\hyperlink{arguments}{}Arguments}{Arguments}}\label{arguments}}

\begin{longtable}[c]{@{}>{\small}p{3cm}>{\raggedright}p{12.5cm}@{}}
\toprule
cl & Character string specifying class of layer (e.g.
\texttt{``ruta\_layer\_dense''}), which will be used to call the
corresponding methods\tabularnewline
\ldots{} & Other parameters (usually \texttt{units},
\texttt{activation})\tabularnewline
\bottomrule
\end{longtable}

\hypertarget{value}{\subsection{\texorpdfstring{\protect\hyperlink{value}{}Value}{Value}}\label{value}}

A construct with class \texttt{``ruta\_layer''}

\hypertarget{examples}{\subsection{\texorpdfstring{\protect\hyperlink{examples}{}Examples}{Examples}}\label{examples}}

\begin{minted}[frame=none]{R}
my_layer <- new_layer("dense", 30, "tanh")

# Equivalent:
my_layer <- dense(30, "tanh")[[1]]
\end{minted}

\section{Sequential network
constructor}\label{sequential-network-constructor}

Source:
\href{https://github.com/fdavidcl/ruta/blob/master/R/network.R}{\texttt{R/network.R}}

\texttt{new\_network.Rd}

Constructor function for networks composed of several sequentially
placed layers. You shouldn't generally need to use this. Instead,
consider concatenating several layers with \texttt{+.ruta\_network}.

\begin{minted}[frame=none]{R}
new_network(...)
\end{minted}

\hypertarget{arguments}{\subsection{\texorpdfstring{\protect\hyperlink{arguments}{}Arguments}{Arguments}}\label{arguments}}

\begin{longtable}[c]{@{}>{\small}p{3cm}>{\raggedright}p{12.5cm}@{}}
\toprule
\ldots{} & Zero or more objects of class
\texttt{``ruta\_layer''}\tabularnewline
\bottomrule
\end{longtable}

\hypertarget{value}{\subsection{\texorpdfstring{\protect\hyperlink{value}{}Value}{Value}}\label{value}}

A construct with class \texttt{``ruta\_network''}

\hypertarget{examples}{\subsection{\texorpdfstring{\protect\hyperlink{examples}{}Examples}{Examples}}\label{examples}}

\begin{minted}[frame=none]{R}
my_network <- new_network(
  new_layer("input", 784, "linear"),
  new_layer("dense",  32, "tanh"),
  new_layer("dense", 784, "sigmoid")
)

# Instead, consider using
my_network <- input() + dense(32, "tanh") + output("sigmoid")
\end{minted}

\section{Noise generator}\label{noise-generator}

Source:
\href{https://github.com/fdavidcl/ruta/blob/master/R/filter.R}{\texttt{R/filter.R}}

\texttt{noise.Rd}

Delegates on noise classes to generate noise of some type

\begin{minted}[frame=none]{R}
noise(type, ...)
\end{minted}

\hypertarget{arguments}{\subsection{\texorpdfstring{\protect\hyperlink{arguments}{}Arguments}{Arguments}}\label{arguments}}

\begin{longtable}[c]{@{}>{\small}p{3cm}>{\raggedright}p{12.5cm}@{}}
\toprule
type & Type of noise, as a character string\tabularnewline
\ldots{} & Parameters for each noise class\tabularnewline
\bottomrule
\end{longtable}

\section{Additive Cauchy noise}\label{additive-cauchy-noise}

Source:
\href{https://github.com/fdavidcl/ruta/blob/master/R/filter.R}{\texttt{R/filter.R}}

\texttt{noise\_cauchy.Rd}

A data filter which adds noise from a Cauchy distribution to instances

\begin{minted}[frame=none]{R}
noise_cauchy(scale = 0.005)
\end{minted}

\hypertarget{arguments}{\subsection{\texorpdfstring{\protect\hyperlink{arguments}{}Arguments}{Arguments}}\label{arguments}}

\begin{longtable}[c]{@{}>{\small}p{3cm}>{\raggedright}p{12.5cm}@{}}
\toprule
scale & Scale for the Cauchy distribution\tabularnewline
\bottomrule
\end{longtable}

\hypertarget{value}{\subsection{\texorpdfstring{\protect\hyperlink{value}{}Value}{Value}}\label{value}}

Object which can be applied to data with \texttt{apply\_filter}

\hypertarget{see-also}{\subsection{\texorpdfstring{\protect\hyperlink{see-also}{}See
also}{See also}}\label{see-also}}

Other noise generators: \texttt{noise\_gaussian}, \texttt{noise\_ones},
\texttt{noise\_saltpepper}, \texttt{noise\_zeros}

\section{Additive Gaussian noise}\label{additive-gaussian-noise}

Source:
\href{https://github.com/fdavidcl/ruta/blob/master/R/filter.R}{\texttt{R/filter.R}}

\texttt{noise\_gaussian.Rd}

A data filter which adds Gaussian noise to instances

\begin{minted}[frame=none]{R}
noise_gaussian(sd = NULL, var = NULL)
\end{minted}

\hypertarget{arguments}{\subsection{\texorpdfstring{\protect\hyperlink{arguments}{}Arguments}{Arguments}}\label{arguments}}

\begin{longtable}[c]{@{}>{\small}p{3cm}>{\raggedright}p{12.5cm}@{}}
\toprule
sd & Standard deviation for the Gaussian distribution\tabularnewline
var & Variance of the Gaussian distribution (optional, only used if
\texttt{sd} is not provided)\tabularnewline
\bottomrule
\end{longtable}

\hypertarget{value}{\subsection{\texorpdfstring{\protect\hyperlink{value}{}Value}{Value}}\label{value}}

Object which can be applied to data with \texttt{apply\_filter}

\hypertarget{see-also}{\subsection{\texorpdfstring{\protect\hyperlink{see-also}{}See
also}{See also}}\label{see-also}}

Other noise generators: \texttt{noise\_cauchy}, \texttt{noise\_ones},
\texttt{noise\_saltpepper}, \texttt{noise\_zeros}

\section{Filter to add ones noise}\label{filter-to-add-ones-noise}

Source:
\href{https://github.com/fdavidcl/ruta/blob/master/R/filter.R}{\texttt{R/filter.R}}

\texttt{noise\_ones.Rd}

A data filter which replaces some values with ones

\begin{minted}[frame=none]{R}
noise_ones(p = 0.05)
\end{minted}

\hypertarget{arguments}{\subsection{\texorpdfstring{\protect\hyperlink{arguments}{}Arguments}{Arguments}}\label{arguments}}

\begin{longtable}[c]{@{}>{\small}p{3cm}>{\raggedright}p{12.5cm}@{}}
\toprule
p & Probability that a feature in an instance is set to
one\tabularnewline
\bottomrule
\end{longtable}

\hypertarget{value}{\subsection{\texorpdfstring{\protect\hyperlink{value}{}Value}{Value}}\label{value}}

Object which can be applied to data with \texttt{apply\_filter}

\hypertarget{see-also}{\subsection{\texorpdfstring{\protect\hyperlink{see-also}{}See
also}{See also}}\label{see-also}}

Other noise generators: \texttt{noise\_cauchy},
\texttt{noise\_gaussian}, \texttt{noise\_saltpepper},
\texttt{noise\_zeros}

\section{Filter to add salt-and-pepper
noise}\label{filter-to-add-salt-and-pepper-noise}

Source:
\href{https://github.com/fdavidcl/ruta/blob/master/R/filter.R}{\texttt{R/filter.R}}

\texttt{noise\_saltpepper.Rd}

A data filter which replaces some values with zeros or ones

\begin{minted}[frame=none]{R}
noise_saltpepper(p = 0.05)
\end{minted}

\hypertarget{arguments}{\subsection{\texorpdfstring{\protect\hyperlink{arguments}{}Arguments}{Arguments}}\label{arguments}}

\begin{longtable}[c]{@{}>{\small}p{3cm}>{\raggedright}p{12.5cm}@{}}
\toprule
p & Probability that a feature in an instance is set to zero or
one\tabularnewline
\bottomrule
\end{longtable}

\hypertarget{value}{\subsection{\texorpdfstring{\protect\hyperlink{value}{}Value}{Value}}\label{value}}

Object which can be applied to data with \texttt{apply\_filter}

\hypertarget{see-also}{\subsection{\texorpdfstring{\protect\hyperlink{see-also}{}See
also}{See also}}\label{see-also}}

Other noise generators: \texttt{noise\_cauchy},
\texttt{noise\_gaussian}, \texttt{noise\_ones}, \texttt{noise\_zeros}

\section{Filter to add zero noise}\label{filter-to-add-zero-noise}

Source:
\href{https://github.com/fdavidcl/ruta/blob/master/R/filter.R}{\texttt{R/filter.R}}

\texttt{noise\_zeros.Rd}

A data filter which replaces some values with zeros

\begin{minted}[frame=none]{R}
noise_zeros(p = 0.05)
\end{minted}

\hypertarget{arguments}{\subsection{\texorpdfstring{\protect\hyperlink{arguments}{}Arguments}{Arguments}}\label{arguments}}

\begin{longtable}[c]{@{}>{\small}p{3cm}>{\raggedright}p{12.5cm}@{}}
\toprule
p & Probability that a feature in an instance is set to
zero\tabularnewline
\bottomrule
\end{longtable}

\hypertarget{value}{\subsection{\texorpdfstring{\protect\hyperlink{value}{}Value}{Value}}\label{value}}

Object which can be applied to data with \texttt{apply\_filter}

\hypertarget{see-also}{\subsection{\texorpdfstring{\protect\hyperlink{see-also}{}See
also}{See also}}\label{see-also}}

Other noise generators: \texttt{noise\_cauchy},
\texttt{noise\_gaussian}, \texttt{noise\_ones},
\texttt{noise\_saltpepper}

\section{Create an output layer}\label{create-an-output-layer}

Source:
\href{https://github.com/fdavidcl/ruta/blob/master/R/layers.R}{\texttt{R/layers.R}}

\texttt{output.Rd}

This layer acts as a placeholder for the output layer in an autoencoder.
The number of units is not needed as it is deduced from the data during
training.

\begin{minted}[frame=none]{R}
output(activation = "linear")
\end{minted}

\hypertarget{arguments}{\subsection{\texorpdfstring{\protect\hyperlink{arguments}{}Arguments}{Arguments}}\label{arguments}}

\begin{longtable}[c]{@{}>{\small}p{3cm}>{\raggedright}p{12.5cm}@{}}
\toprule
activation & Optional, string indicating activation function (linear by
default)\tabularnewline
\bottomrule
\end{longtable}

\hypertarget{value}{\subsection{\texorpdfstring{\protect\hyperlink{value}{}Value}{Value}}\label{value}}

A construct with class \texttt{``ruta\_network''}

\hypertarget{see-also}{\subsection{\texorpdfstring{\protect\hyperlink{see-also}{}See
also}{See also}}\label{see-also}}

Other neural layers: \texttt{dense}, \texttt{dropout}, \texttt{input},
\texttt{layer\_keras}, \texttt{variational\_block}

\section{Draw a neural network}\label{draw-a-neural-network}

Source:
\href{https://github.com/fdavidcl/ruta/blob/master/R/network_plot.R}{\texttt{R/network\_plot.R}}

\texttt{plot.ruta\_network.Rd}

Draw a neural network

\begin{minted}[frame=none]{R}
# S3 method for ruta_network
plot(x, ...)
\end{minted}

\hypertarget{arguments}{\subsection{\texorpdfstring{\protect\hyperlink{arguments}{}Arguments}{Arguments}}\label{arguments}}

\begin{longtable}[c]{@{}>{\small}p{3cm}>{\raggedright}p{12.5cm}@{}}
\toprule
\begin{minipage}[t]{0.08\columnwidth}\raggedright\strut
x
\strut\end{minipage} &
\begin{minipage}[t]{0.73\columnwidth}\raggedright\strut
A \texttt{``ruta\_network''} object
\strut\end{minipage}\tabularnewline
\begin{minipage}[t]{0.08\columnwidth}\raggedright\strut
\ldots{}
\strut\end{minipage} &
\begin{minipage}[t]{0.73\columnwidth}\raggedright\strut
Additional parameters for style. Available parameters:
\strut\end{minipage}\tabularnewline
\begin{minipage}[t]{0.08\columnwidth}\raggedright\strut
\strut\end{minipage} &
\begin{minipage}[t]{0.73\columnwidth}\raggedright\strut
- \texttt{bg}: Color for the text over layers
\strut\end{minipage}\tabularnewline
\begin{minipage}[t]{0.08\columnwidth}\raggedright\strut
\strut\end{minipage} &
\begin{minipage}[t]{0.73\columnwidth}\raggedright\strut
- \texttt{fg}: Color for the background of layers
\strut\end{minipage}\tabularnewline
\begin{minipage}[t]{0.08\columnwidth}\raggedright\strut
\strut\end{minipage} &
\begin{minipage}[t]{0.73\columnwidth}\raggedright\strut
- \texttt{log}: Use logarithmic scale
\strut\end{minipage}\tabularnewline
\bottomrule
\end{longtable}

\hypertarget{examples}{\subsection{\texorpdfstring{\protect\hyperlink{examples}{}Examples}{Examples}}\label{examples}}

\begin{minted}[frame=none]{R}
net <-
  input() +
  dense(1000, "relu") + dropout() +
  dense(100, "tanh") +
  dense(1000, "relu") + dropout() +
  output("sigmoid")
plot(net, log = TRUE, fg = "#30707a", bg = "#e0e6ea")
\end{minted}

\section{Inspect Ruta objects}\label{inspect-ruta-objects}

Source:
\href{https://github.com/fdavidcl/ruta/blob/master/R/autoencoder.R}{\texttt{R/autoencoder.R}},
\href{https://github.com/fdavidcl/ruta/blob/master/R/loss.R}{\texttt{R/loss.R}},
\href{https://github.com/fdavidcl/ruta/blob/master/R/network.R}{\texttt{R/network.R}}

\texttt{print-methods.Rd}

Inspect Ruta objects

\begin{minted}[frame=none]{R}
# S3 method for ruta_autoencoder
print(x, ...)

# S3 method for ruta_loss_named
print(x, ...)

# S3 method for ruta_loss
print(x, ...)

# S3 method for ruta_network
print(x, ...)
\end{minted}

\hypertarget{arguments}{\subsection{\texorpdfstring{\protect\hyperlink{arguments}{}Arguments}{Arguments}}\label{arguments}}

\begin{longtable}[c]{@{}>{\small}p{3cm}>{\raggedright}p{12.5cm}@{}}
\toprule
x & An object\tabularnewline
\ldots{} & Unused\tabularnewline
\bottomrule
\end{longtable}

\hypertarget{value}{\subsection{\texorpdfstring{\protect\hyperlink{value}{}Value}{Value}}\label{value}}

Invisibly returns the same object passed as parameter

\hypertarget{examples}{\subsection{\texorpdfstring{\protect\hyperlink{examples}{}Examples}{Examples}}\label{examples}}

\begin{minted}[frame=none]{R}
print(autoencoder(c(256, 10), loss = correntropy()))
#> Autoencoder learner
#> ----------------------------------------
#> Type: robust 
#> 
#> Network structure:
#>  input
#>  dense(256 units) - linear
#>  dense(10 units) - linear
#>  dense(256 units) - linear
#>  dense - linear
#> 
#> Loss: correntropy 
#> ----------------------------------------
\end{minted}

\section{Retrieve reconstructions for input
data}\label{retrieve-reconstructions-for-input-data}

Source:
\href{https://github.com/fdavidcl/ruta/blob/master/R/autoencoder.R}{\texttt{R/autoencoder.R}}

\texttt{reconstruct.Rd}

Extracts the reconstructions calculated by a trained autoencoder for the
specified input data after encoding and decoding. \texttt{predict} is an
alias for \texttt{reconstruct}.

\begin{minted}[frame=none]{R}
reconstruct(learner, data)

# S3 method for ruta_autoencoder
predict(object, ...)
\end{minted}

\hypertarget{arguments}{\subsection{\texorpdfstring{\protect\hyperlink{arguments}{}Arguments}{Arguments}}\label{arguments}}

\begin{longtable}[c]{@{}>{\small}p{3cm}>{\raggedright}p{12.5cm}@{}}
\toprule
learner & Trained autoencoder model\tabularnewline
data & data.frame to be passed through the network\tabularnewline
object & Trained autoencoder model\tabularnewline
\ldots{} & Rest of parameters, unused\tabularnewline
\bottomrule
\end{longtable}

\hypertarget{value}{\subsection{\texorpdfstring{\protect\hyperlink{value}{}Value}{Value}}\label{value}}

Matrix containing the reconstructions

\hypertarget{see-also}{\subsection{\texorpdfstring{\protect\hyperlink{see-also}{}See
also}{See also}}\label{see-also}}

\texttt{encode}, \texttt{decode}

\section{Sparsity regularization}\label{sparsity-regularization}

Source:
\href{https://github.com/fdavidcl/ruta/blob/master/R/autoencoder_sparse.R}{\texttt{R/autoencoder\_sparse.R}}

\texttt{sparsity.Rd}

Sparsity regularization

\begin{minted}[frame=none]{R}
sparsity(high_probability, weight)
\end{minted}

\hypertarget{arguments}{\subsection{\texorpdfstring{\protect\hyperlink{arguments}{}Arguments}{Arguments}}\label{arguments}}

\begin{longtable}[c]{@{}>{\small}p{3cm}>{\raggedright}p{12.5cm}@{}}
\toprule
high\_probability & Expected probability of the high value of the
encoding layer. Set this to a value near zero in order to minimize
activations in that layer.\tabularnewline
weight & The weight of the sparsity regularization\tabularnewline
\bottomrule
\end{longtable}

\hypertarget{value}{\subsection{\texorpdfstring{\protect\hyperlink{value}{}Value}{Value}}\label{value}}

A Ruta regularizer object for the sparsity, to be inserted in the
encoding layer.

\hypertarget{references}{\subsection{\texorpdfstring{\protect\hyperlink{references}{}References}{References}}\label{references}}

\begin{itemize}
\item
  \href{http://papers.nips.cc/paper/3313-sparse-deep-belief-net-model-for-visual-area-v2}{Sparse
  deep belief net model for visual area V2}
\item
  Andrew Ng, Sparse Autoencoder.
  \href{https://web.stanford.edu/class/cs294a/sparseAutoencoder_2011new.pdf}{CS294A
  Lecture Notes}
\end{itemize}

\hypertarget{see-also}{\subsection{\texorpdfstring{\protect\hyperlink{see-also}{}See
also}{See also}}\label{see-also}}

\texttt{autoencoder\_sparse}, \texttt{make\_sparse}, \texttt{is\_sparse}

\section{Access subnetworks of a
network}\label{access-subnetworks-of-a-network}

Source:
\href{https://github.com/fdavidcl/ruta/blob/master/R/network.R}{\texttt{R/network.R}}

\texttt{sub-.ruta\_network.Rd}

Access subnetworks of a network

\begin{minted}[frame=none]{R}
# S3 method for ruta_network
[(net, index)
\end{minted}

\hypertarget{arguments}{\subsection{\texorpdfstring{\protect\hyperlink{arguments}{}Arguments}{Arguments}}\label{arguments}}

\begin{longtable}[c]{@{}>{\small}p{3cm}>{\raggedright}p{12.5cm}@{}}
\toprule
net & A \texttt{``ruta\_network''} object\tabularnewline
index & An integer vector of indices of layers to be
extracted\tabularnewline
\bottomrule
\end{longtable}

\hypertarget{value}{\subsection{\texorpdfstring{\protect\hyperlink{value}{}Value}{Value}}\label{value}}

A \texttt{``ruta\_network''} object containing the specified layers.

\hypertarget{examples}{\subsection{\texorpdfstring{\protect\hyperlink{examples}{}Examples}{Examples}}\label{examples}}

\begin{minted}[frame=none]{R}
(input() + dense(30))[2]
#> Network structure:
#>  dense(30 units) - linearlong <- input() + dense(1000) + dense(100) + dense(1000) + output()
short <- long[c(1, 3, 5)]
\end{minted}

\section{Convert a Ruta object onto Keras objects and
functions}\label{convert-a-ruta-object-onto-keras-objects-and-functions}

Source:
\href{https://github.com/fdavidcl/ruta/blob/master/R/generics.R}{\texttt{R/generics.R}}

\texttt{to\_keras.Rd}

Generic function which uses the Keras API to build objects out of Ruta
wrappers

\begin{minted}[frame=none]{R}
to_keras(x, ...)
\end{minted}

\hypertarget{arguments}{\subsection{\texorpdfstring{\protect\hyperlink{arguments}{}Arguments}{Arguments}}\label{arguments}}

\begin{longtable}[c]{@{}>{\small}p{3cm}>{\raggedright}p{12.5cm}@{}}
\toprule
x & Object to be converted\tabularnewline
\ldots{} & Remaining parameters depending on the method\tabularnewline
\bottomrule
\end{longtable}

\section{Extract Keras models from an autoencoder
wrapper}\label{extract-keras-models-from-an-autoencoder-wrapper}

Source:
\href{https://github.com/fdavidcl/ruta/blob/master/R/autoencoder.R}{\texttt{R/autoencoder.R}},
\href{https://github.com/fdavidcl/ruta/blob/master/R/autoencoder_variational.R}{\texttt{R/autoencoder\_variational.R}}

\texttt{to\_keras.ruta\_autoencoder.Rd}

Extract Keras models from an autoencoder wrapper

\begin{minted}[frame=none]{R}
# S3 method for ruta_autoencoder
to_keras(learner, encoder_end = "encoding",
  decoder_start = "encoding", weights_file = NULL)

# S3 method for ruta_autoencoder_variational
to_keras(learner, ...)
\end{minted}

\hypertarget{arguments}{\subsection{\texorpdfstring{\protect\hyperlink{arguments}{}Arguments}{Arguments}}\label{arguments}}

\begin{longtable}[c]{@{}>{\small}p{3cm}>{\raggedright}p{12.5cm}@{}}
\toprule
learner & Object of class \texttt{``ruta\_autoencoder''}. Needs to have a
member \texttt{input\_shape} indicating the number of attributes of the
input data\tabularnewline
encoder\_end & Name of the Keras layer where the encoder
ends\tabularnewline
decoder\_start & Name of the Keras layer where the decoder
starts\tabularnewline
weights\_file & The name of a hdf5 weights file in order to load from a
trained model\tabularnewline
\ldots{} & Additional parameters for
\texttt{to\_keras.ruta\_autoencoder}\tabularnewline
\bottomrule
\end{longtable}

\hypertarget{value}{\subsection{\texorpdfstring{\protect\hyperlink{value}{}Value}{Value}}\label{value}}

A list with several Keras models:

\begin{itemize}
\item
  \texttt{autoencoder}: model from the input layer to the output layer
\item
  \texttt{encoder}: model from the input layer to the encoding layer
\item
  \texttt{decoder}: model from the encoding layer to the output layer
\end{itemize}

\hypertarget{see-also}{\subsection{\texorpdfstring{\protect\hyperlink{see-also}{}See
also}{See also}}\label{see-also}}

\texttt{autoencoder}

\section{Convert Ruta layers onto Keras
layers}\label{convert-ruta-layers-onto-keras-layers}

Source:
\href{https://github.com/fdavidcl/ruta/blob/master/R/layers.R}{\texttt{R/layers.R}}

\texttt{to\_keras.ruta\_layer\_input.Rd}

Convert Ruta layers onto Keras layers

\begin{minted}[frame=none]{R}
# S3 method for ruta_layer_input
to_keras(x, input_shape, ...)

# S3 method for ruta_layer_dense
to_keras(x, input_shape,
  model = keras::keras_model_sequential(), ...)

# S3 method for ruta_layer_custom
to_keras(x, input_shape,
  model = keras::keras_model_sequential(), ...)
\end{minted}

\hypertarget{arguments}{\subsection{\texorpdfstring{\protect\hyperlink{arguments}{}Arguments}{Arguments}}\label{arguments}}

\begin{longtable}[c]{@{}>{\small}p{3cm}>{\raggedright}p{12.5cm}@{}}
\toprule
x & The layer object\tabularnewline
input\_shape & Number of features in training data\tabularnewline
\ldots{} & Unused\tabularnewline
model & Keras model where the layer will be added\tabularnewline
\bottomrule
\end{longtable}

\hypertarget{value}{\subsection{\texorpdfstring{\protect\hyperlink{value}{}Value}{Value}}\label{value}}

A Layer object from Keras

\section{Obtain a Keras block of layers for the variational
autoencoder}\label{obtain-a-keras-block-of-layers-for-the-variational-autoencoder}

Source:
\href{https://github.com/fdavidcl/ruta/blob/master/R/autoencoder_variational.R}{\texttt{R/autoencoder\_variational.R}}

\texttt{to\_keras.ruta\_layer\_variational.Rd}

This block contains two dense layers representing the mean and log var
of a Gaussian distribution and a sampling layer.

\begin{minted}[frame=none]{R}
# S3 method for ruta_layer_variational
to_keras(x, input_shape,
  model = keras::keras_model_sequential(), ...)
\end{minted}

\hypertarget{arguments}{\subsection{\texorpdfstring{\protect\hyperlink{arguments}{}Arguments}{Arguments}}\label{arguments}}

\begin{longtable}[c]{@{}>{\small}p{3cm}>{\raggedright}p{12.5cm}@{}}
\toprule
x & The layer object\tabularnewline
input\_shape & Number of features in training data\tabularnewline
model & Keras model where the layers will be added\tabularnewline
\ldots{} & Unused\tabularnewline
\bottomrule
\end{longtable}

\hypertarget{value}{\subsection{\texorpdfstring{\protect\hyperlink{value}{}Value}{Value}}\label{value}}

A Layer object from Keras

\hypertarget{references}{\subsection{\texorpdfstring{\protect\hyperlink{references}{}References}{References}}\label{references}}

\begin{itemize}
\item
  \href{https://arxiv.org/abs/1312.6114}{Auto-Encoding Variational
  Bayes}
\item
  \href{http://blog.fastforwardlabs.com/2016/08/22/under-the-hood-of-the-variational-autoencoder-in.html}{Under
  the Hood of the Variational Autoencoder (in Prose and Code)}
\item
  \href{https://keras.rstudio.com/articles/examples/variational_autoencoder.html}{Keras
  example: Variational autoencoder}
\end{itemize}

\section{Obtain a Keras loss}\label{obtain-a-keras-loss}

Source:
\href{https://github.com/fdavidcl/ruta/blob/master/R/autoencoder_contractive.R}{\texttt{R/autoencoder\_contractive.R}},
\href{https://github.com/fdavidcl/ruta/blob/master/R/autoencoder_robust.R}{\texttt{R/autoencoder\_robust.R}},
\href{https://github.com/fdavidcl/ruta/blob/master/R/autoencoder_variational.R}{\texttt{R/autoencoder\_variational.R}},
and 2 more

\texttt{to\_keras.ruta\_loss\_named.Rd}

Builds the Keras loss function corresponding to a name

\begin{minted}[frame=none]{R}
# S3 method for ruta_loss_contraction
to_keras(x, learner, ...)

# S3 method for ruta_loss_correntropy
to_keras(x, ...)

# S3 method for ruta_loss_variational
to_keras(x, learner, ...)

# S3 method for ruta_loss_named
to_keras(x, ...)
\end{minted}

\hypertarget{arguments}{\subsection{\texorpdfstring{\protect\hyperlink{arguments}{}Arguments}{Arguments}}\label{arguments}}

\begin{longtable}[c]{@{}>{\small}p{3cm}>{\raggedright}p{12.5cm}@{}}
\toprule
x & A \texttt{``ruta\_loss\_named''} object\tabularnewline
learner & The learner object including the keras model which will use
the loss function\tabularnewline
\ldots{} & Rest of parameters, ignored\tabularnewline
\bottomrule
\end{longtable}

\hypertarget{value}{\subsection{\texorpdfstring{\protect\hyperlink{value}{}Value}{Value}}\label{value}}

A function which returns the corresponding loss for given true and
predicted values

\hypertarget{references}{\subsection{\texorpdfstring{\protect\hyperlink{references}{}References}{References}}\label{references}}

\begin{itemize}
\tightlist
\item
  Contractive loss:
  \href{https://wiseodd.github.io/techblog/2016/12/05/contractive-autoencoder/}{Deriving
  Contractive Autoencoder and Implementing it in Keras}
\end{itemize}

\begin{itemize}
\tightlist
\item
  Correntropy loss:
  \href{https://ieeexplore.ieee.org/abstract/document/6854900/}{Robust
  feature learning by stacked autoencoder with maximum correntropy
  criterion}
\end{itemize}

\begin{itemize}
\item
  Variational loss:

  \begin{itemize}
  \item
    \href{https://arxiv.org/abs/1312.6114}{Auto-Encoding Variational
    Bayes}
  \item
    \href{http://blog.fastforwardlabs.com/2016/08/22/under-the-hood-of-the-variational-autoencoder-in.html}{Under
    the Hood of the Variational Autoencoder (in Prose and Code)}
  \item
    \href{https://keras.rstudio.com/articles/examples/variational_autoencoder.html}{Keras
    example: Variational autoencoder}
  \end{itemize}
\end{itemize}

\section{Build a Keras network}\label{build-a-keras-network}

Source:
\href{https://github.com/fdavidcl/ruta/blob/master/R/network.R}{\texttt{R/network.R}}

\texttt{to\_keras.ruta\_network.Rd}

Build a Keras network

\begin{minted}[frame=none]{R}
# S3 method for ruta_network
to_keras(x, input_shape)
\end{minted}

\hypertarget{arguments}{\subsection{\texorpdfstring{\protect\hyperlink{arguments}{}Arguments}{Arguments}}\label{arguments}}

\begin{longtable}[c]{@{}>{\small}p{3cm}>{\raggedright}p{12.5cm}@{}}
\toprule
x & A \texttt{``ruta\_network''} object\tabularnewline
input\_shape & The length of each input vector (number of input
attributes)\tabularnewline
\bottomrule
\end{longtable}

\hypertarget{value}{\subsection{\texorpdfstring{\protect\hyperlink{value}{}Value}{Value}}\label{value}}

A list of Keras Tensor objects with an attribute \texttt{``encoding''}
indicating the index of the encoding layer

\section{Translate sparsity regularization to Keras
regularizer}\label{translate-sparsity-regularization-to-keras-regularizer}

Source:
\href{https://github.com/fdavidcl/ruta/blob/master/R/autoencoder_sparse.R}{\texttt{R/autoencoder\_sparse.R}}

\texttt{to\_keras.ruta\_sparsity.Rd}

Translate sparsity regularization to Keras regularizer

\begin{minted}[frame=none]{R}
# S3 method for ruta_sparsity
to_keras(x, activation)
\end{minted}

\hypertarget{arguments}{\subsection{\texorpdfstring{\protect\hyperlink{arguments}{}Arguments}{Arguments}}\label{arguments}}

\begin{longtable}[c]{@{}>{\small}p{3cm}>{\raggedright}p{12.5cm}@{}}
\toprule
x & Sparsity object\tabularnewline
activation & Name of the activation function used in the encoding
layer\tabularnewline
\bottomrule
\end{longtable}

\hypertarget{value}{\subsection{\texorpdfstring{\protect\hyperlink{value}{}Value}{Value}}\label{value}}

Function which can be used as activity regularizer in a Keras layer

\hypertarget{references}{\subsection{\texorpdfstring{\protect\hyperlink{references}{}References}{References}}\label{references}}

\begin{itemize}
\item
  \href{http://papers.nips.cc/paper/3313-sparse-deep-belief-net-model-for-visual-area-v2}{Sparse
  deep belief net model for visual area V2}
\item
  Andrew Ng, Sparse Autoencoder.
  \href{https://web.stanford.edu/class/cs294a/sparseAutoencoder_2011new.pdf}{CS294A
  Lecture Notes} (2011)
\end{itemize}

\section{Obtain a Keras weight decay}\label{obtain-a-keras-weight-decay}

Source:
\href{https://github.com/fdavidcl/ruta/blob/master/R/autoencoder_weight_decay.R}{\texttt{R/autoencoder\_weight\_decay.R}}

\texttt{to\_keras.ruta\_weight\_decay.Rd}

Builds the Keras regularizer corresponding to the weight decay

\begin{minted}[frame=none]{R}
# S3 method for ruta_weight_decay
to_keras(x, ...)
\end{minted}

\hypertarget{arguments}{\subsection{\texorpdfstring{\protect\hyperlink{arguments}{}Arguments}{Arguments}}\label{arguments}}

\begin{longtable}[c]{@{}>{\small}p{3cm}>{\raggedright}p{12.5cm}@{}}
\toprule
x & A \texttt{``ruta\_weight\_decay''} object\tabularnewline
\ldots{} & Rest of parameters, ignored\tabularnewline
\bottomrule
\end{longtable}

\section{Train a learner object with
data}\label{train-a-learner-object-with-data}

Source:
\href{https://github.com/fdavidcl/ruta/blob/master/R/autoencoder.R}{\texttt{R/autoencoder.R}},
\href{https://github.com/fdavidcl/ruta/blob/master/R/generics.R}{\texttt{R/generics.R}}

\texttt{train.ruta\_autoencoder.Rd}

This function compiles the neural network described by the learner
object and trains it with the input data.

\begin{minted}[frame=none]{R}
# S3 method for ruta_autoencoder
train(learner, data, validation_data = NULL,
  metrics = NULL, epochs = 20, optimizer = keras::optimizer_rmsprop(),
  ...)

train(learner, ...)
\end{minted}

\hypertarget{arguments}{\subsection{\texorpdfstring{\protect\hyperlink{arguments}{}Arguments}{Arguments}}\label{arguments}}

\begin{longtable}[c]{@{}>{\small}p{3cm}>{\raggedright}p{12.5cm}@{}}
\toprule
\begin{minipage}[t]{0.09\columnwidth}\raggedright\strut
learner
\strut\end{minipage} &
\begin{minipage}[t]{0.84\columnwidth}\raggedright\strut
A \texttt{``ruta\_autoencoder''} object
\strut\end{minipage}\tabularnewline
\begin{minipage}[t]{0.09\columnwidth}\raggedright\strut
data
\strut\end{minipage} &
\begin{minipage}[t]{0.84\columnwidth}\raggedright\strut
Training data: columns are attributes and rows are instances
\strut\end{minipage}\tabularnewline
\begin{minipage}[t]{0.09\columnwidth}\raggedright\strut
validation\_data
\strut\end{minipage} &
\begin{minipage}[t]{0.84\columnwidth}\raggedright\strut
Additional numeric data matrix which will not be used for training but
the loss measure and any metrics will be computed against it
\strut\end{minipage}\tabularnewline
\begin{minipage}[t]{0.09\columnwidth}\raggedright\strut
metrics
\strut\end{minipage} &
\begin{minipage}[t]{0.84\columnwidth}\raggedright\strut
Optional list of metrics which will evaluate the model but won't be
optimized. See \texttt{keras::compile}
\strut\end{minipage}\tabularnewline
\begin{minipage}[t]{0.09\columnwidth}\raggedright\strut
epochs
\strut\end{minipage} &
\begin{minipage}[t]{0.84\columnwidth}\raggedright\strut
The number of times data will pass through the network
\strut\end{minipage}\tabularnewline
\begin{minipage}[t]{0.09\columnwidth}\raggedright\strut
optimizer
\strut\end{minipage} &
\begin{minipage}[t]{0.84\columnwidth}\raggedright\strut
The optimizer to be used in order to train the model, can be any
optimizer object defined by Keras (e.g.
\texttt{keras::optimizer\_adam()})
\strut\end{minipage}\tabularnewline
\begin{minipage}[t]{0.09\columnwidth}\raggedright\strut
\ldots{}
\strut\end{minipage} &
\begin{minipage}[t]{0.84\columnwidth}\raggedright\strut
Additional parameters for \texttt{keras::fit}. Some useful parameters:
\strut\end{minipage}\tabularnewline
\begin{minipage}[t]{0.09\columnwidth}\raggedright\strut
\strut\end{minipage} &
\begin{minipage}[t]{0.84\columnwidth}\raggedright\strut
- \texttt{batch\_size} The number of examples to be grouped for each
gradient update. Use a smaller batch size for more frequent weight
updates or a larger one for faster optimization.
\strut\end{minipage}\tabularnewline
\begin{minipage}[t]{0.09\columnwidth}\raggedright\strut
\strut\end{minipage} &
\begin{minipage}[t]{0.84\columnwidth}\raggedright\strut
- \texttt{shuffle} Whether to shuffle the training data before each
epoch, defaults to \texttt{TRUE}
\strut\end{minipage}\tabularnewline
\bottomrule
\end{longtable}

\hypertarget{value}{\subsection{\texorpdfstring{\protect\hyperlink{value}{}Value}{Value}}\label{value}}

Same autoencoder passed as parameter, with trained internal models

\hypertarget{see-also}{\subsection{\texorpdfstring{\protect\hyperlink{see-also}{}See
also}{See also}}\label{see-also}}

\texttt{autoencoder}

\hypertarget{examples}{\subsection{\texorpdfstring{\protect\hyperlink{examples}{}Examples}{Examples}}\label{examples}}

\begin{minted}[frame=none]{R}
# Minimal example ================================================
iris_model <- train(autoencoder(2), as.matrix(iris[, 1:4]))
# Simple example with MNIST ======================================
library(keras)

# Load and normalize MNIST
mnist = dataset_mnist()
x_train <- array_reshape(
  mnist$train$x, c(dim(mnist$train$x)[1], 784)
)
x_train <- x_train / 255.0
x_test <- array_reshape(
  mnist$test$x, c(dim(mnist$test$x)[1], 784)
)
x_test <- x_test / 255.0

# Autoencoder with layers: 784-256-36-256-784
learner <- autoencoder(c(256, 36), "binary_crossentropy")
train(
  learner,
  x_train,
  epochs = 1,
  optimizer = "rmsprop",
  batch_size = 64,
  validation_data = x_test,
  metrics = list("binary_accuracy")
)
\end{minted}

\section{Create a variational block of
layers}\label{create-a-variational-block-of-layers}

Source:
\href{https://github.com/fdavidcl/ruta/blob/master/R/autoencoder_variational.R}{\texttt{R/autoencoder\_variational.R}}

\texttt{variational\_block.Rd}

This variational block consists in two dense layers which take as input
the previous layer and a sampling layer. More specifically, these layers
aim to represent the mean and the log variance of the learned
distribution in a variational autoencoder.

\begin{minted}[frame=none]{R}
variational_block(units, epsilon_std = 1, seed = NULL)
\end{minted}

\hypertarget{arguments}{\subsection{\texorpdfstring{\protect\hyperlink{arguments}{}Arguments}{Arguments}}\label{arguments}}

\begin{longtable}[c]{@{}>{\small}p{3cm}>{\raggedright}p{12.5cm}@{}}
\toprule
units & Number of units\tabularnewline
epsilon\_std & Standard deviation for the normal distribution used for
sampling\tabularnewline
seed & A seed for the random number generator. \textbf{Setting a seed is
required if you want to save the model and be able to load it
correctly}\tabularnewline
\bottomrule
\end{longtable}

\hypertarget{value}{\subsection{\texorpdfstring{\protect\hyperlink{value}{}Value}{Value}}\label{value}}

A construct with class \texttt{``ruta\_layer''}

\hypertarget{see-also}{\subsection{\texorpdfstring{\protect\hyperlink{see-also}{}See
also}{See also}}\label{see-also}}

\texttt{autoencoder\_variational}

Other neural layers: \texttt{dense}, \texttt{dropout}, \texttt{input},
\texttt{layer\_keras}, \texttt{output}

\hypertarget{examples}{\subsection{\texorpdfstring{\protect\hyperlink{examples}{}Examples}{Examples}}\label{examples}}

\begin{minted}[frame=none]{R}
variational_block(3)
#> Network structure:
#>  variational(3 units)
\end{minted}

\section{Weight decay}\label{weight-decay}

Source:
\href{https://github.com/fdavidcl/ruta/blob/master/R/autoencoder_weight_decay.R}{\texttt{R/autoencoder\_weight\_decay.R}}

\texttt{weight\_decay.Rd}

A wrapper that describes a weight decay regularization of the encoding
layer

\begin{minted}[frame=none]{R}
weight_decay(decay = 0.02)
\end{minted}

\hypertarget{arguments}{\subsection{\texorpdfstring{\protect\hyperlink{arguments}{}Arguments}{Arguments}}\label{arguments}}

\begin{longtable}[c]{@{}>{\small}p{3cm}>{\raggedright}p{12.5cm}@{}}
\toprule
decay & Numeric value indicating the amount of decay\tabularnewline
\bottomrule
\end{longtable}

\hypertarget{value}{\subsection{\texorpdfstring{\protect\hyperlink{value}{}Value}{Value}}\label{value}}

A regularizer object containing the set parameters

\section{Save and load Ruta models}\label{save-and-load-ruta-models}

Source:
\href{https://github.com/fdavidcl/ruta/blob/master/R/save.R}{\texttt{R/save.R}}

\texttt{save\_as.Rd}

Functions to save a trained or untrained Ruta learner into a file and
load it

\begin{minted}[frame=none]{R}
save_as(learner, file = paste0(substitute(learner), ".tar.gz"), dir,
  compression = "gzip")

load_from(file)
\end{minted}

\hypertarget{arguments}{\subsection{\texorpdfstring{\protect\hyperlink{arguments}{}Arguments}{Arguments}}\label{arguments}}

\begin{longtable}[c]{@{}>{\small}p{3cm}>{\raggedright}p{12.5cm}@{}}
\toprule
learner & The \texttt{``ruta\_autoencoder''} object to be
saved\tabularnewline
file & In \texttt{save}, filename with extension (usually
\texttt{.tar.gz}) where the object will be saved. In \texttt{load}, path
to the saved model\tabularnewline
dir & Directory where to save the file. Use \texttt{``.''} to save in the
current working directory or \texttt{tempdir()} to use a temporary
one\tabularnewline
compression & Type of compression to be used, for R function
\texttt{tar}\tabularnewline
\bottomrule
\end{longtable}

\hypertarget{value}{\subsection{\texorpdfstring{\protect\hyperlink{value}{}Value}{Value}}\label{value}}

\texttt{save\_as} returns the filename where the model has been saved,
\texttt{load\_from} returns the loaded model as a
\texttt{``ruta\_autoencoder''} object

\hypertarget{examples}{\subsection{\texorpdfstring{\protect\hyperlink{examples}{}Examples}{Examples}}\label{examples}}

\begin{minted}[frame=none]{R}
library(purrr)

x <- as.matrix(iris[, 1:4])# Save a trained model
saved_file <-
  autoencoder(2) %>%
  train(x) %>%
  save_as("my_model.tar.gz", dir = tempdir())

# Load and use the model
encoded <- load_from(saved_file) %>% encode(x)
#> Loading weights from /tmp/RtmphVC8m6/ruta/weights.hdf5
\end{minted}
